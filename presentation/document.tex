\documentclass{beamer}
\usetheme{default}

\title{S3 performance em HPC}
\author{João Pedro Abreu de Souza}
\begin{document}
\begin{frame}[plain]
    \maketitle
\end{frame}
\begin{frame}{HPC}
	\begin{itemize}
		\item Processamento
		\pause
		\item Memoria
		\pause
		\item IO (foco do artigo)
	\end{itemize}
\end{frame}
\begin{frame}{IO - verificação}
	\begin{itemize}
		\item IO500
		\item MD-Workbench
	\end{itemize}
\end{frame}
\begin{frame}{Armazenamento local}
	\begin{itemize}
		\item Sistemas de arquivos
		\begin{itemize}
			\item ext4
			\item btfs
			\item zfs
		\end{itemize}
	\end{itemize}
\end{frame}
\begin{frame}{Armazenamento distribuido}
	\begin{itemize}
		\item Objetos
\begin{itemize}
	\item s3
\end{itemize}
		\item Sistema de arquivos
		\begin{itemize}
			\item Ceph
			\item OpenStack Swift
			\item Minio
			\item MPI-IO
		\end{itemize}
		\end{itemize}
\end{frame}
\begin{frame}{Bibliotecas s3 em hpc}
	\begin{itemize}
		\item libs3
		\item S3Remote
		\item S3Embedded
	\end{itemize}
\end{frame}
\begin{frame}{Conclusão}
	\begin{itemize}
		\item Bibliotecas s3 não estão prontas para hpc
		\item Com o IO-500 aumentado, a comunidade pode observar o s3
		\item integrações entre S3 e sistemas de arquivos permitiriam que clientes s3 acessassem IO de HPC
	\end{itemize}
\end{frame}
\end{document}
